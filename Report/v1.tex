It is important to note that the searches in these experiments were conducted in a single-story structure with a relatively simple geometry. Occupants were only required to be moved about 6 m (20 ft) to be extracted from the structure. If the floor plan were larger or more complicated, the times required to find and remove occupants likely would have been longer. Following suppression, the rate of gas concentration decrease was not as high as the rate of temperature decrease as described in Section \ref{subsec:ff_int}. Thus, although the FED rate was decreasing in the time that the search crew took to find and remove the occupants, the occupants were still exposed to high concentrations of toxic gases. Because of the nature of the governing equations, even though the hazard was decreasing, the FED of any occupant would continue to increase until they were removed from the structure. This is demonstrated in Figure \ref{fig:vic_removal}, which plots the increase in FED toxicity from the time that the search crew made entry to the time that they located the dining room occupant. To control for variation in search crew effectiveness and variation in fire dynamics, the FED was calculated at the search time for all search groups and applied to each experiment. The increase in FED toxicity from the time of search team entry to the time of victim location was calculated for the 11 experiments with data for the dining room victim. The box and whisker plots show the distribution of the FED increase for each experiment, with the whiskers representing the FED increase corresponding to the maximum and minimum victim location times, the box corresponding to the middle quartile of increase, and the red line represents the FED increase for the median search time.  The chart shows that as the time to find the simulated occupant increases, the tenability of the occupant is affected dramatically. The time between the fastest and slowest victim removal was 46 seconds, yet in some experiments this time difference could result in a 60\% greater difference in FED increase. It follows that once the victim is found, the toxicity exposure to the victim will continue to increase during the removal process, as long as the occupant is still breathing. This emphasizes the importance of rapid removal of occupants located during the search in an effort to minimize the toxic exposure of these occupants. 

 % Note that because there were no local gas concentration measurements for the simulated occupants, the stationary sample point in the dining room was assumed to represent the toxic exposure to the occupant for the duration of the removal process. While it is likely that the gas concentrations and resultant FED exposures would be different within the occupant removal path, the dining room concentrations offers an approximation to the toxic insult in order to explore this phenomenon.

\begin{figure}[!ht]
	\centering
	\includegraphics[width=.75\textwidth]{../Figures/victim_removal/V1}
	\caption[Relationship between occupant location time and increase in FED between entry of search team and location of dining room occupant]{Relationship between occupant location time and \% increase in FED between entry of search team and location of dining room occupant. Whiskers represent highest and lowest times to occupant found, red lines represent the FED increases of the middle two quartiles of the location times, and red line indicates FED increase corresponding to median location time. }
	\label{fig:vic_removal}
\end{figure}

In this series of experiments, most of the simulated occupants were removed from the structure out the front door from the location that they were found. The shortest occupant removal time, however, was observed in Experiment 4, where the search crew removed the occupant out of the rear door of the near closed bedroom. This removal method exposed the occupant to toxic gases for the shortest duration, but also avoided dragging the occupant through the hallway and living room, where the  concentrations of products of combustion were higher than in the closed bedroom. Thus, depending on the conditions within the structure, the location of the occupant within the structure, and the knowledge of the search company of alternative means of egress, the ideal path for occupant removal may be out of an opening separate from the one which the search team entered through, such as a rear or side door, or even through a window or down a ladder. this emphasizes the importance of situational awareness among the members of the search company and coordination of occupant removal. 