\documentclass[12pt,oneside]{book}
\usepackage{times,mathptmx}
\usepackage[pdftex]{graphicx}
\usepackage{calc}
\usepackage{tabularx,ragged2e,booktabs,caption,subcaption}
\usepackage{array}
\newcolumntype{L}[1]{>{\raggedright\let\newline\\\arraybackslash\hspace{0pt}}m{#1}}
\newcolumntype{C}[1]{>{\centering\let\newline\\\arraybackslash\hspace{0pt}}m{#1}}
\newcolumntype{R}[1]{>{\raggedleft\let\newline\\\arraybackslash\hspace{0pt}}m{#1}}
\usepackage{multirow}
\usepackage{tocloft}
\usepackage{xcolor}
\usepackage{color,soul}
\usepackage{amsmath}
\definecolor{linknavy}{rgb}{0,0,0.50196}
\definecolor{linkred}{rgb}{1,0,0}
\definecolor{linkblue}{rgb}{0,0,1}
\usepackage{float}
\usepackage{graphpap}
\usepackage{rotating}
\usepackage{graphicx}
\usepackage{geometry}
\usepackage{relsize}
\usepackage{ltablex}
\usepackage{longtable}
\usepackage{lscape}
\usepackage{amssymb}
\usepackage{makeidx} % Create index at end of document
\usepackage[nottoc,notlof,notlot]{tocbibind} % Put the bibliography and index in the ToC
\usepackage{lastpage} % Automatic last page number reference.
\usepackage[T1]{fontenc}
\usepackage{enumerate}
\usepackage{upquote}
\usepackage{moreverb}
\usepackage{xfrac}
\usepackage{cite}
\usepackage{tikz}
% \usepackage{subfig}
% \usepackage{caption}
\usepackage[toc,page]{appendix}
\usepackage{notoccite}

\usepackage{titlesec}
\titleformat{\chapter}[hang] 
{\normalfont\huge\bfseries}{\chaptertitlename\ \thechapter}{1em}{} 

\newcommand{\nopart}{\expandafter\def\csname Parent-1\endcsname{}} % To fix table of contents in pdf.

\usepackage{siunitx}
\sisetup{
    detect-all = true,
    input-decimal-markers = {.},
    input-ignore = {,},
    inter-unit-product = \ensuremath{{}\cdot{}},
    multi-part-units = repeat,
    number-unit-product = \text{~},
    per-mode = fraction,
    separate-uncertainty = true,
}

\usepackage{listings}
\usepackage{textcomp}
\definecolor{lbcolor}{rgb}{0.96,0.96,0.96}

\usepackage[pdftex,
        colorlinks=true,
        urlcolor=linkblue,     % \href{...}{...} external (URL)
        citecolor=linkred,     % citation number colors
        linkcolor=linknavy,    % \ref{...} and \pageref{...}
        pdfproducer={pdflatex},
        pdfpagemode=UseNone,
        bookmarksopen=true,
        plainpages=false,
        verbose]{hyperref}

\setlength{\textwidth}{6.5in}
\setlength{\textheight}{9.0in}
\setlength{\topmargin}{0.in}
\setlength{\headheight}{0.pt}
\setlength{\headsep}{0.in}
\setlength{\parindent}{0.0in}
\setlength{\itemindent}{0.25in}
\setlength{\oddsidemargin}{0.0in}
\setlength{\evensidemargin}{0.0in}
% \setlength{\leftmargini}{\parindent} % Controls the indenting of the "bullets" in a list
\setlength{\cftsecnumwidth}{0.45in}
\setlength{\cftsubsecnumwidth}{0.5in}
\setlength{\cftfignumwidth}{0.45in}
\setlength{\cfttabnumwidth}{0.45in}
\setlength{\parskip}{1em}

\newcommand{\titlesigs}
{
\large
\flushright{UL Firefighter Safety Research Institute\\
{\em Stephen Kerber, Director} \\
\hspace{1in} \\
}
}

\newcommand{\headerB}[1]{
\flushleft{
\fontsize{28}{33.6}\selectfont
\bf{#1}
}
}

\newcommand{\headerC}[1]{
\vspace{.5in}
\flushright{\fontsize{14}{16.8}\selectfont
#1}
}

% \newcolumntype{L}{>{\centering\arraybackslash}m{4cm}}

\floatstyle{boxed}
\newfloat{notebox}{H}{lon}
\newfloat{warning}{H}{low}

\newenvironment{conditions}
  {\par\vspace{\abovedisplayskip}\noindent\begin{tabular}{>{$}l<{$} @{${}={}$} l}}
  {\end{tabular}\par\vspace{\belowdisplayskip}}


% Rename chapter headings
\renewcommand{\chaptername}{}
\renewcommand{\bibname}{References}

\usepackage{fancyhdr}
\usepackage{placeins}
\pagestyle{fancy}
\lhead{}
\rhead{}
\chead{}
\renewcommand{\headrulewidth}{0pt}

% UN-COMMENT TO PLACE WATERMARK
\usepackage{draftwatermark}
\SetWatermarkText{DRAFT}
\SetWatermarkScale{1}

\usepackage{subcaption}
\usepackage{xfrac}

\begin{document}

\frontmatter

\begin{center}
\large{Residential Structure Fire Intervention and Victim Tenability}\\
\normalsize{$^1$Steve Kerber, $^2$Gavin P. Horn, $^3$Kenneth W Fent, $^{2,4}$Denise L Smith, $^1$John Regan \\}
\end{center}

\flushright{$^1$ UL Firefighter Safety Research Institute; Columbia, MD, USA \\
$^2$ University of Illinois, Fire Service Institute; Urbana-Champaign, IL, USA\\
$^3$ National Institute for Occupational Safety \& Health; Cincinnati, OH, USA \\
$^4$ Skidmore College; Saratoga Springs, NY }

\vspace{2in}

\flushleft{Address correspondence to:\\
Steve Kerber\\
UL Firefighter Safety Research Institute \\
6200 Old Dobbin Lane, Suite 150\\
Columbia, MD 21045\\
USA\\

Telephone: 847.664.6999\\
E-mail: stephen.kerber@ul.com}
	
\vspace{1.5in}


\flushleft{Acknowledgements/Funding: This work was supported by the Department of Homeland Security Fire Prevention and Safety Grant \#EMW-2013-FP-00766. \\

Disclosure: There are no conflicts of interest regarding this work.}




\vfill

% \flushright{\includegraphics[width=2.in]{../8_Images/FSRI_GraphicShield}}

\titlesigs
\newpage
\section{Abstract}
\mainmatter
\flushleft
\section{Introduction}
\label{sec:intro}

A primary goal of firefighting is to extinguish the fire to protect life and property. While this basic goal may seem obvious and straightforward to a civilian, the tactics used by the fire department to accomplish this goal may vary considerably. Based on an accumulating body of evidence, many fire departments are emphasizing getting water on the fire as soon as possible to improve conditions inside the structure (Kerber, 2013). Such an approach is often called a ``transitional'' attack in which firefighters apply water through a window to initially suppress the fire before they enter the building to completely extinguish the fire and ensure there is no further fire growth.  This approach contrasts with many departments that have been taught that it is best to enter the house through the front door with a charged hose line. In theory, the goal of this ``interior'' fire attack is to find the seat of the fire and extinguish it as soon as possible to protect potential victims. To date, there is no research that has considered the effect of different firefighting tactics on the firefighter's physiological responses to their work and the impact on potential occupant survivability.
This paper will focus on occupant survivability and the interaction between firefighter suppression (transitional attack and direct interior attack) and search and rescue tactics on occupant survivability.  This will be done by utilizing building temperature and gas concentration measurements and local temperature measurements affixed to two simulated occupants.  

\section{Methods}
\label{sec:methods}

\subsection{Subjects}
Participants were recruited through a nationwide multimedia effort along with a focused effort by a statewide network of firefighters who teach and train at the Illinois Fire Service Institute's (IFSI) Champaign campus. Forty (n=40) firefighters (36 male, 4 female) from departments in Illinois, Georgia, Indiana, Ohio, South Dakota and Wisconsin participated in this study. The firefighters were 37.6$\pm$8.9 years old, 1.80$\pm$0.08m tall, weighed 89.8$\pm$14.5 kg and had an average BMI of 27.6$\pm$3.4 kg/m2.
All participants were required to have completed a medical evaluation consistent with National Fire Protection Association (NFPA) 1582 in the past 12 months. We recruited relatively experienced firefighters who had up to date training, could complete the assigned tasks as directed, and were familiar with live-fire policies and procedures. Throughout the study protocol, all firefighters were required to wear their self-contained breathing apparatus (SCBA) prior to entering the structure. The research team supplied all personal protective equipment (PPE) for the participants to enhance standardization and to ensure that all protective equipment adhered to NFPA standards.

\subsection{Study Design}
Teams of 12 firefighters were deployed to suppress fires in a realistic firefighting scenario that involved a multiple-room fire (two separate bedrooms) in a 111 m$^2$ residential structure. Each team of 12 firefighters worked in pairs to perform six different job assignments that included operations on the inside of the structure during active fire (fire attack and search \& rescue), on the outside of the structure during active fire (command \& pump operator and outside ventilation), and to conduct overhaul operations after the fire had been suppressed (firefighters searched for smoldering items and removed items from the structure).  The job assignments are described in Table 1. 
In all, 12 different trials were conducted (one per day) each with twelve firefighters as described above.  The firefighters responded to two scenarios that differed only in the tactics used by the Inside Attack team: (a) traditional interior attack from the “unburned side” (advancement through the front door to extinguish the fire) and (b) transitional attack (water applied into the bedroom fires through an exterior window - from the “unburned side” - prior to advancing through the front door to extinguish the fire).  The firefighters performed the same role using both tactics, then were reassigned to different job assignments and performed another two scenarios – again using the same two tactics on separate days.  While most firefighters attended four sessions of the study (n=31), a small group were only available for two sessions (n=9) and one firefighter withdrew from the study and wasn't replaced until after the first two scenarios.

\subsection{Study Protocol}
Following recruitment, subjects completed all required paperwork and anthropomorphic measurements (height, weight) were collected.  Firefighters received a core temperature pill that they ingested 6–12 hours prior to data collection. Upon arrival on each day, firefighters were instrumented with skin temperature patches on the back of their neck and upper arm that they wore throughout the trial. Multiple pre- and post-firefighting cardiovascular measurements and chemical exposure samples (biological and PPE) were collected prior to the initiation of the live fire evaluation (these data will be reported elsewhere).  The firefighter subjects were then deployed to complete their firefighting work in a purpose-built live-fire research test structure. 
In order to safely and reliably conduct this study, a structure was designed and built to have all of the interior finishes and features of a single family dwelling, yet contained specialized safety systems and hardened construction techniques that ensured participants' safety. The house was based on a design by a residential architectural company to be representative of a home constructed in the mid-twentieth century with walls and doorways separating all of the rooms and 2.4 m ceilings. The home had an approximate floor area of 111 m2, with 8 total rooms, including 4 bedrooms and 1 bathroom (closed off during experiments). Interior finishes in the burn rooms were protected by 15.9 mm Type X gypsum board on the ceiling and 12.7 mm gypsum board on the walls. To maximize the use of the structure and minimize time between experiments, the house was mirrored so that there were 2 bedrooms on each side where the fires were ignited.  During each experiment a temporary wall was constructed at the end of the hallway to isolate 2 bedrooms so that they could be repaired and readied for the next experiment.  
Furniture was acquired from a single source such that each room was furnished identically (same item, manufacture, make model and layout of all furnishings) for all 12 experiments. The bedrooms, where the fires were ignited, were furnished with a double bed (covered with a foam mattress topper, comforter and pillow), stuffed chair, side table, lamp, dresser and flat screen television.  The floors were covered with polyurethane foam padding and polyester carpet.  All other rooms of the structure were also furnished to provide obstacles for the firefighter, but those furnishings were not involved in the fire. Figure 1 provides a rendering of the structure with the roof cut away to show the interior layout with furniture and floor coverings.  The tan floor shows the carpet placement and the grey floor shows the cement floor or simulated tile locations.  
Fires were ignited in the stuffed chair in Bedrooms 1 \& 2 (labeled Bedrooms 5 \& 6 for the mirrored configuration) using a remote ignition device and a book of matches to create a small flaming ignition source.  The flaming fire was allowed to grow until temperatures in the fire rooms reached levels determined to be near peak values based on pilot studies (i.e. room had ‘flashed over').  When interior temperatures of both fire rooms exceeded 600 $^{\circ}$C at the ceiling, the fire department dispatch was simulated and firefighters responded by walking approximately 16 meters from the data collection bay to the front of the structure.  The time of dispatch was between 4 and 5 minutes after ignition for all 12 experiments.  

\subsection{Measures}

\subsubsection{Building Temperature Measurements}
To assess fire dynamics throughout the fire scenarios, measurements included gas temperature, gas concentrations, pressure, heat flux, thermal imaging, and video recording. Detailed measurement locations can be found in Figure 1.  
Gas temperature was measured with bare-bead, ChromelAlumel (type K) thermocouples with a 0.5 mm nominal diameter. Thermocouple arrays were located in every room. The thermocouple locations in the living room, dining room, hallway, Bedroom 4, and kitchen had an array of thermocouples with measurement locations of 0.3 m, 0.6 m, 0.9 m, 1.2 m, 1.5 m, 1.8 m and 2.1 m above the floor. The thermocouple locations in Bedroom 1/5, Bedroom 2/6, and Bedroom 3 had an array of thermocouples with measurement locations of 0.3 m, 0.9 m, 1.5 m, and 2.1 m above the floor. 

\subsubsection{Building Gas Concentration Measurements}
Gas concentrations of oxygen, carbon monoxide, and carbon dioxide were measured (OxyMat 6 and Ultramat 23 NDIR; Siemens) at 0.9 m from the floor inside and outside of Bedrooms 3 and 4.   These measurement locations are adjacent to the simulated occupant sitting outside of Bedroom 4 (Bedroom 3 when mirrored configuration was used) and the simulated occupant laying on the bed in Bedroom 3 (Bedroom 4 when mirrored configuration was used).  This measurement location is also consistent with a potential occupant crawling to escape the fire.  The uncertainty of the measured concentration is 1\% of the maximum concentration measurement. The maximum concentration measurements were 5\% for CO and 20\% for CO2. The gases were extracted from the corners of rooms so as to minimize risk of damage from removing burned fuels and touring visitors. All data was collected at a frequency of 1 Hz. 

\subsubsection{Firefighter Intervention Measurements}
For each scenario, firefighter intervention was monitored and recorded utilizing standard video cameras placed outside and throughout the structure.  Thermal imaging cameras were also placed inside the structure to examine firefighter movements and simulated occupant search and rescue tactics.  Portable cameras were also affixed to the simulated occupants to qualitatively capture their exposure and movements from their locations to the outside of the structure as the firefighters rescued them.
\subsection{Occupant Tenability}
Occupant tenability, which is the survivability of occupants in the fire environment, is a primary concern for any firefighting operation.  Two standard measures of occupant tenability were used during these experiments - temperature and gas concentration - based upon the fractional effective dose methodology (FED) from ISO 13571 [15]. This methodology provides a method to calculate the time to incapacitation based on an accumulated exposure to either toxic gases:
\begin{equation}\label{eqn:FED_co}FED_{CO} = \Sigma [\frac{\phi_{CO}}{3.5}*\nu_{CO_2}*\Delta t]\end{equation}

\begin{equation}\label{eqn:FED_vco2}\nu_{CO_2}=exp(\frac{\phi_{CO_2}}{5})\end{equation}
Or local ambient air temperature 

\begin{equation}\label{eqn:FED_temp}FED_{temp}=\Sigma[\frac{T^{3.61}}{4.1*10^8}*\Delta t]\end{equation}

where $\nu_{CO_2}$ is a frequency factor to account for the increased rate of breathing due to carbon dioxide, $\phi_{CO_2}$ and $\phi_{CO}$ are the mole fractions (\%) of carbon dioxide and carbon monoxide, T is the temperature near the occupant ($^{\circ}$C), and $\Delta t$ is the time increment of the measurements made in the experiments in minutes (1/60 in these experiments).  According to ISO 13571, the uncertainty in Eq. 1 is $\pm$ 20\% and the uncertainty in Eq. 2 is $\pm$ 35\%.  Equation 3 only applies for temperatures greater than 120$^{\circ}$C, which is taken as the lower limit to this method.  
FED relates to the probability of the conditions being non-tenable for a certain percent of the population through a lognormal distribution.  For reference, FED = 0.3 is the criterion used to determine the time of incapacitation for susceptible individuals (young children, elderly, and/or unhealthy occupants) and corresponds to untenability for 11\% of the population, and FED = 1.0 is the value at which 50\% of the population would experience untenable conditions.

FED's were calculated at an elevation of 0.9 m above the floor, representative of a relative worst case scenario of a person crawling on the floor. The time to exceed the thresholds for all of the experiments in each house for both heat (only convection considered) and carbon monoxide/carbon dioxide are calculated inside and outside of Bedrooms 3 and 4.  For consistency, since the structure was mirrored between experiments the measurement locations will be referred to as near hall, far hall, near bedroom and far bedroom, referenced to their location with respect to where the fire was ignited.  For example when Bedrooms 1 and 2 are ignited bedroom 3 is the near bedroom but when Bedroom 5 and 6 are ignited Bedroom 3 is the far bedroom.  It should be noted that the values assume the simulated occupant was in that location for the duration of the experiment. These estimates may be considered lower bound scenarios as additional thermal risks may be present from exposure to large radiant heat exposures or from the additive effects of exposure to a variety of different fireground gases such as HCN.  

FED's were also calculated utilizing the portable temperature and gas measurement devices placed on the heads of the two simulated occupants.  This allowed for the closest estimate possible of what an occupant would be exposed to, especially as they were manipulated and removed from the structure by the firefighters.   

It is true though that both heat exposure and toxic gas exposure will increase with increasing height in the structure. This should be kept in mind if considering an occupant walking out of the structure and will result in even higher FED values and lower times to untenability than at the 0.9 m height. However, the focus of this study is on the tenability at the crawling height of an occupant.


\subsection{Statistical Analysis}

\section{Results}

\subsection{Building Temperature Measurements}
The 0.91 m ( 3 ft.) temperature was used to assess the thermal exposure to which an occupant trapped at different locations within the structure may be exposed. The average temperature in the 30 seconds prior to firefighter intervention in the hallway, outside of the fire rooms was 320.0$\pm$64.2$^{\circ}$C. In the dining room, remote from the seat of the fire, the average temperature was 135.5$\pm$34.8$^{\circ}$C. The standard deviation was 20\% and 25\% of the mean temperature for hallway and dining room locations, respectively. When compared to the combined instrument uncertainty of 15\%, the temperatures at the time of firefighter intervention are within a reasonable margin of uncertainty. The 0.91 m (3 ft.) temperatures measured in the closed bedrooms were significantly;y lower than those measured in the areas of the structure open to the fire. The average temperature in the 30 seconds prior to suppression was 23.0$\pm$1.6$^{\circ}$C in the near bedroom and 20.6$\pm$1.2$^{\circ}$C in the far bedroom. The standard deviation for these sensors are 7.0\% and 5.8\% of the average temperatures for the near and far bedrooms, respectively, less than the 15\% combined uncertainty.

The 0.91 m ( 3 ft.) temperature in open areas of the structure decreased substantially in the 60 seconds following suppression. The temperature decrease was considered in a 60 second window following firefighter intervention because this time frame encompasses water application into both windows for the transitional attack scenario and the attack team to reach the hallway and apply water to both fire rooms for the interior attack scenarios.  In the hallway between the fire rooms, this temperature decrease was 261$\pm$101$^{\circ}$C for the transitional attack scenarios and 313$\pm$69$^{\circ}$C for the interior attack scenarios. The temeprature decrease was not significantly different between the two attack scenarios (p=0.42). The maximum rate of decrease, however, occurred more quickly after suppression for the transitional attack (8.4$\pm$3.9 s) than for the interior attack (33.0$\pm$8.3 s). This is likely because the limited visibility and geometry hinders the interior attack, an obstacle which is not present in the transitional attack. 

\subsection{Building Gas Concentration Measurements}
At the time of firefighter intervention, the FED calculations varied considerably between experiments. At the near hall location, just outside of the bedroom fires,  the mean FED value at firefighter intervention was 1.67$\pm$1.85. In the far hall location, next to the first simulated occupant, the mean FED was 0.30$\pm$0.35. The variation that was noted in these measurements can be attributed to variations in the carbon monoxide, carbon dioxide, and oxygen measurements. The variation in these measurements was greater than the uncertainty of these sensors. Additionally, because the FED equations presented in Equations \ref{eqn:FED_co} and \ref{eqn:FED_vco2} are exponential in nature, small measurement variations will result in larger variations in the FED calculation. Further, ISO 1371 [ref] lists the uncertainty for the FED calculations as high as 35\%. In the closed bedrooms, the FED magnitude at the time of firefighter intervention was negligible at the time of firefighter intervention. 

After firefighter intervention, the FED values continued to increase in magnitude. Table \ref{tab:final_fed} lists the maximum FEDs observed for each measurement location in each experiment. The mean total FED values for the open gas locations were 4.53$\pm$3.91 and 1.88$\pm$1.26 for the near hall and far hall locations, respectively. These values were substantially higher than the FEDs recorded in the closed bedrooms, which were 0.41$\pm$0.43 and 0.66$\pm$.51 for the near and far closed bedrooms, respectively. This indicates that the closed bedroom door is an effective barrier to products of combustion, which are noted in high concentrations low to the floor in the open areas of the house. Even in the bedroom closest to the seat of the fire in the bedroom, the conditions behind the closed door are significantly lower (p=0.005) than in the hallway immediately outside the bedroom. The near bedroom sample point was also significantly lower (p=0.003) than the sample point located in the dining room, next to the open victim.

\begin{table}[!ht]
    \centering
    \caption{Final Gas FED Values at Each Measurement Location}
    \label{tab:final_fed}
    \begin{tabular}{ccccc}
    \toprule[1.5pt]
	\textbf{Experiment}  &   \textbf{Near Hall}& \textbf{Dining Room}& \textbf{Near Closed Bedroom}& \textbf{Far Closed Bedroom} \\ 
	  \midrule                                                                   
	Exp. 1 &       n.a&     2.88&         0.34&        0.12 \\  
	Exp. 2 &         2&      n.a&         0.28&        0.78 \\
	Exp. 3 &     12.19&     3.82&         1.19&        0.36 \\               
	Exp. 4 &      2.31&     0.66&         0.08&        1.97 \\                
	Exp. 5 &      5.53&     1.12&          1.4&        0.76 \\                 
	Exp. 6 &      1.81&     1.18&         0.28&        0.71 \\                 
	Exp. 7 &      7.04&     2.29&         0.57&        0.58 \\                
	Exp. 8 &      1.95&     1.26&         0.41&        1.03 \\            
	Exp. 9 &      3.38&     1.63&         0.21&        0.72 \\              
	Exp. 10&      1.36&      1.2&         0.06&         n.a \\         
	Exp. 11&     12.26&     4.63&          n.a&        0.85 \\             
	Exp. 12&      4.57&     1.79&         0.03&         n.a \\           
	 \bottomrule[1.25pt] 

    \end{tabular}
     \flushleft{n.a indicates a sensor malfunction at that location }
\end{table}


\subsection{Firefighter Intervention Measurements}
The timeline of firefighter interventions varied with both the method of attack (transitional vs. interior) and the actions taken by the subjects during their execution of the fireground. The search crew of two firefighters were instructed to search the structure beginning in the half of the house opposite the structure. As the crews searched, they found the first victim, located in the corner of the dining room, propped against the far bedroom door. Once they removed this victim, they continued their search pattern through the far closed bedroom, the kitchen, and the living room, before reaching the closed bedroom closer to the fire bedrooms, where the second victim was located. In Experiments 1 and 5, the search crews missed the far closed bedroom, and the door was never opened. Table \ref{tab:victim_times} shows the mean times for the search team to find and remove each victim. The method of attack was found to not have a significant difference on the time required to find the dining room victim (p=0.75) or the bedroom victim (p=0.32). Similarly, time required to remove the dining room victim (p=0.38) and bedroom victim (p=0.85) was not found to be significantly different between attack methods. 

\begin{table}[!ht]
    \centering
    \caption{Times to Find and Remove Victims}
    \label{tab:victim_times}
    \begin{tabular}{ccccc}
    \toprule[1.5pt]
 	 Event&								Interior Attack Time (s)&	Transitional Attack Time (s)\\
 	\midrule 
  	Time to find dining room victim&	35.7$\pm$15.1&				38.3$\pm$9.8\\
  	Time to remove dining room victim&	42.0$\pm$21.2&				54.3$\pm$21.4\\
  	Time to find bedroom victim&		218.3$\pm$61.9&				263.7$\pm$74.1\\
  	Time to remove bedroom victim&		49.8$\pm$63.5&				55.5$\pm$21.1\\
 	\bottomrule[1.25pt] 
    \end{tabular}
\end{table}

As the search company opened the doors to the near and far bedrooms in order to gain access and complete their search, the bedroom was no longer isolated from the rest of the structure. In the cases where the remote bedroom was opened and searched, an increase in FED rate was observed as products of combustion filled the room. The magnitude of this increase varied between tests, and is likely dependent on the amount of these gases that were already present in the room at the time of opening, and the conditions outside of the room at this time.  For the four tests in which the door was not opened during the initial part of the search, the FED rates during the latter parts of the test were among the lowest that were observed during the tests. 

When considering the impact of suppression on victim tenability within the structure, Experiments 3 (transitional) and 5 (interior) were treated as outliers, and neglected from the comparisons. In Experiment 3, the attack crew applied water for only 4 seconds in each window, which allowed the fire to rebound by the time that the interior crews entered the structures. In Experiment 5, when the attack crew reached the hallway, they did not have a sufficient length of hose to apply water into the fire rooms, reducing the effectiveness of the attack. 

For the other experiments, after water was applied, whether from the interior or the exterior, the FED rate in open areas of the structure began to decrease. For the gas sample location in the hallway outside of the fire rooms, this inflection point occurred 43.5$\pm$29.3 seconds from the time that water was first applied for the transitional attack experiments and 38.2$\pm$33.1 seconds from the time that the attack crew made entry for the interior attack experiments. For the gas sample location in the dining room, this inflection point occurred 101.6$\pm$44.8 seconds from the time that water was first applied for the transitional attack experiments and 27.0$\pm$24.6 seconds from the time that the attack crew made entry for the interior attack experiments. Apart from the two outliers experiments discussed previously, the FED rate did not increase for the rest of the experiment following water application. Thus, this FED rate inflection point can be taken as the time at which conditions would start to improve for victims in an areas of the structure not isolated by a closed door or other barrier. For the near hall position, there was no significant difference between attack methods, but for the dining room location, the interior attack method did improve conditions significantly faster than the transitional attack method (p=0.03). This may be because during the interior attack, the opening of the front door provides an access route for fresh air to enter the structure and hot gases to exit. The entrainment of fresh air, combined with the water application of the attack team, may be responsible for the more rapid decrease in fED rate. in the transitional attack, the opening of the front door is delayed until the attack team has repositioned, so the positive effects of suppression take longer.

The immediate positive effect that was noted in the open areas was not observed in the closed bedrooms, however. Just as the closed bedroom doors isolate the interior of the bedroom, delaying the entry of products of combustion into the rooms, they delay the positive effects of suppression. For each experiment, the FED rate continued to increase after suppression in each of the closed bedrooms. The inflection point for conditions in the closed bedroom closer to the fire was 456.3$\pm$95.1 for the transitional attack experiments and 328.0$\pm$63.9 for the interior attack experiments (p=0.10).  The inflection point for conditions in the closed bedroom closer to the fire was 452.3$\pm$97.0 for the transitional attack experiments and 413.5$\pm$128 for the interior attack experiments (p=0.70).  While the FED rate takes longer to decrease in the closed bedrooms than in the open bedrooms, the magnitude of the rate is far smaller than the rate in open areas, even after the bedroom door has been opened. Because the bedroom doors were not opened until after suppression had started, and thus the FED rate in areas open to the fire had already started to decrease, these bedrooms were not subject to the high rates of FED increase noted in those open areas, as they may have been if the doors had been opened prior to suppression. 


\section{Discussion}


\subsection{Building Temperature Measurements}



\subsection{Building Gas Concentration Measurements}

\subsection{Firefighter Intervention Measurements}

\section{Conclusions}
\end{document}